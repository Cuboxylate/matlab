\documentclass{journal}
\usepackage[margin=2cm]{geometry}
\title{COSC1001 Question 1\\ written responses}
\author{Denbeigh Stevens - 312079885\\ Sebastian Dunn - 310178916\\ University of Sydney}
\begin{document}
\begin{flushleft}
\maketitle
\subsection*{Question 1c}
The produced histograms tend to have a larger amount toward (100,100), because that is the initial starting position of the ants.
This is not dissimilar to a 3D normal plot with centre 100, over length 200.

As we increase the number of ants, this becomes more pronounced, and the density of the ants becomes a lot larger.

\subsection*{Question 1d}
The function plot results in a shape similar to a circle, showing a continuous regular growth outwards as the number of ants is increased. At higher numbers, the shape becomes closer to a circle, showing how an aggregation of random changes on a micro-scale can lead to continuous behaviours on a macro-scale.

\end{flushleft}
\end{document}
